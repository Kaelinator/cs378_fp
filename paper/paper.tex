\documentclass[twocolumn]{article}
\usepackage{color}
\usepackage[top=1in,bottom=1in,left=1.2in,right=1.2in]{geometry}
\usepackage{hyperref}
\usepackage[small]{titlesec}

\newcommand{\todo}[1]{\textcolor{cyan}{\textbf{TODO:} #1}}

\title{CS378: Final Project - Data Artifacts \\ \url{https://github.com/Kaelinator/cs378_fp}}
\author{Kael Kirk \\ krk2563}

\date{\today}

\begin{document}
\maketitle


\begin{abstract}
  Many models have been created to solve the question answerting task for short
  to medium lengthed examples. 
\end{abstract}

\section{Introduction (5pt)}

\todo{This section should include three paragraphs. The first paragraph is to briefly describe task and data. The second paragraph should describe the results of your analysis, and the third paragraph describing your fix and main experimental take aways. }


\section{Task/Dataset/Model Description (15pt)}

\todo{Describe the task/dataset/model you are working on. Clearly define your task with mathematical notations. Describe your learning algorithm. You must formally specify the loss function, stopping criteria, training data, etc used for the model you are analyzing in the next section. Remember, every notation you use must be defined.}

\section{Performance Analysis (25pt)}

\todo{Describe clearly how you conducted your analysis of the model performance, and report the outcomes. You should provide (1) some examples of both \textbf{specific errors/behavior} from the model and (2) analysis or discussion of \textbf{the general class of mistakes the model makes}. For characterizing the general class, try to come up with rules that can identify a set of challenging examples (e.g., ``examples containing \emph{not}'') and try to visualize in charts, graphs, or tables what you believe to be relevant statistics about the data. For example, how frequent is the class of examples where such error pattern apply? This part of the report should be at least one page.}


\section{Describing Your Fix (20pt)}



\todo{You should describe modifications you have made to improve performances. We will evaluate based on how reasonable is your fix  -- if your fix and motivation does not link, your modificiation is unreasonable. For example, perhaps you tried to modify neural network training in a way that is totally unconnected to your stated goal, or your modification was erroneous. Describe the baseline model to be compared to your approach as well here. Again, clearly define all notations. Someone who is reading your report (with reasonable background, e.g., your classmates in this course) should have a reasonable idea how to implement it themselves. Describe hyperparameters, including how they are selected, if you have hyperparameters involved. 
}

\section{Evaluating Your Fix (25pt)}
\todo{
Your writeup should address how effective is your fix, how broadly applicable is your fix, etc. Providing a single number (overall accuracy) is necessary but not sufficient here. For example, if your change made the model better on challenging NLI examples, you could try to quantify that on one or more slices of the data, give examples of predictions that are fixed, or even use model interpretation techniques to try to support claims about how your improved model is doing its ``reasoning.'' (You can look at the papers listed above to get a sense of how to do such fine-grained evaluation).  You should report results from a baseline approach (your initial trained model) as well as your ``best'' method. If doing your own project, baselines such as majority class, random, or a linear classifier are important to see. \textbf{Ablations}: If you tried several things, analyze the contribution from each one. These should be \emph{minimal} changes to the same system; try running things with just one aspect different in order to assess how important that aspect is. This part of the report should be at least one page.}


% \todo{Describe the data you use, including how many examples are in the training, development, and test sets. Please also provide shallow statistics of your data, with at least the vocabulary size and instruction length. It is best to report all the statistics, including counts, in a table. Please describe how you compute statistics that can be computed in different ways (i.e., vocabulary size). Describe how you pre-process the data, including how you treat casing, tokenization, unknown words, and anything else that you did to the raw data before providing it to your model. If you are using a subset of the data, explain why and discuss tradeoffs (you will also need to back them with experiments).}

% \section{Implementation Details (3pt)}

% \todo{Briefly describe the implementation details. No need to copy details already specified in the assignment. Include any hyper-parameters the model has. If there are any optimizations that you introduced, this is the place to describe them. Did you have to do any special optimizations to make the method work in reasonable time/memory? Describe it here.}

% \section{}

 \section{Related Work (5pt)}
\todo{Briefly discuss prior research papers related to your approach. This will likely some papers in the project description document. How is your approach different from existing studies?}
% \paragraph{Ablations (15pt)}

% \todo{Please report your ablation studies on the development set. Make sure to ablate each feature set independently to show they contribute to your final result.  If a feature set does not improve performance, please do not include it -- it doesn't help. You can still discuss it, but there is really no point in including it in your final model. Reporting the results must be done in a table, and each result must be discussed in the text.}

% \section{Error Analysis (7pt)}

% \todo{Qualitative analysis of selected failure examples. Show and discuss error examples from your development set. You must identify certain classes of errors and use the examples to illustrate them. This is often best to show in a table.}

\section{Conclusion (5pt)}
\todo{Brief conclusion summarizing findings from both numerical results and qualitative analysis.}


\end{document}

